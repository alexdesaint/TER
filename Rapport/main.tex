\documentclass[french]{article}

\usepackage[utf8]{inputenc}
\usepackage[siunitx]{circuitikz}%circuits élèc
\usetikzlibrary{babel}%compatibilité babel

\usepackage{forloop}%Boucles
\usepackage{textcomp}%Flèches

\usepackage[T1]{fontenc}%tous les laractères
\usepackage[french]{babel}%guillement
\usepackage{listings}%ajout de code

\usepackage{fancyhdr}

\usepackage{tocbibind}

\begin{document}


\begin{titlepage}
	\centering
    
    \includegraphics[width=6cm]{photoAlex/logo_upssitech.png}
    \par
    
    \vspace{1cm}
    
	{\Large Rapport des Travaux Encadrés de Recherche\par}
	\vspace{1.5cm}
	{\huge\bfseries Machine à temps de vol pour trampoline\par}
	\vspace{2cm}
	{\Large\itshape
    Alexandre de Lingua de Saint Blanquat

	Mohamed El-Mourabit

	Hugo Diringer

	Hippolyte Catteau

	François Mahé
    \par}
    
	\vfill
    Cliente :\par
    Gaëlle Robert

	\vfill
	{\large 3 février 2019\par}
\end{titlepage}

\begin{center}
\section*{Remerciement}
    
    Nous voudrions remercier tout particulièrement Gaëlle Robert de nous avoir permis de travailler sur ce projet passionnant et aussi pour son aide et sa disponibilité.
    
    
    \vspace{2cm}
    
    Ce projet à été fait pour l'association C.T. GYM de Toulouse,
    
    \makebox[\textwidth]{\includegraphics[width=4cm]{photoAlex/logo_gym}}
    
    et encadré par les enseignants de l'Université Paul Sabatier.
    
    \makebox[\textwidth]{\includegraphics[width=10cm]{photoAlex/logo_UPS}}
\end{center}

\newpage

%Introduction
\section*{Introduction}
Durant la deuxième année à l’Upssitech, nous devons réaliser un projet nommé « TER » (Travaux Encadrés de Recherche). Ce projet a pour but de nous familiariser avec la gestion de projet et la recherche à propos de ce dernier.


Nous avons donc eu un sujet concernant une machine à temps de vol pour trampoline proposer par Mme Gaëlle Robert, entraineur de sport de trampoline au club de Toulouse Rangueil. En effet, quand on rentre dans la compétition au trampoline, une part importante de la note concerne le temps de vol c’est-à-dire le temps que l’athlète passe en l’air. Lors de ces dix sauts sur lequel il est évalué, son temps total passé en l’air (en seconde) est ajouté à son score total donc une partie non négligeable. Le problème du client était que les modèles de machine à temps de vol proposé seulement par des Russes et des Canadiens atteignaient des prix qui n’étaient pas accessibles pour un petit club comme celui de Rangueil. Notre but principal était donc de réaliser cette machine à temps de vol à des prix dits « low cost ». 


Le projet partant de zéro, nous avons dû effectuer toutes les tâches de la réalisation d’un produit. Nous avons commencé par la réalisation d’un cahier des charges avec le client puis d’une phase de conception suivie d’une phase de développement et enfin une phase de test pour redéfinir les problèmes qui nous ont échappé pendant la conception.


Nous avons fini par rendre le produit fonctionnel, satisfaisant le cahier des charges dans tous ses points
\newpage

%table des matière
\tableofcontents
\newpage

%Cahier des charges
\section{Cahier des charges}
Pour la réalisation du projet, nous sommes partis de rien. Il a fallu donc établir un cahier des charges afin de bien définir les besoins du client. Nous nous sommes donc rendus sur place au gymnase de Rangueil pour rencontrer le client et prendre toutes les informations nécessaires et voir les types de trampoline que nous allions utiliser. Notre cahier des charges s’est donc composé au fur et à mesure de trois parties. 

Une partie concernant la machine à temps de vol qui devait respecter les règles de compétition de trampoline émises par la fédération française de trampoline et qui contenait :
\begin{itemize}
    \item Un chronomètre précis au millième de secondes près
    \item Une led qui s’allume chaque fois que le trampoliniste arrive sur la toile pour qu’un juge vérifie que le produit fonctionne correctement
    \item Un bouton permettant de démarrer le chronomètre et enregistrer donc les dix valeurs de sauts
\end{itemize}

La deuxième partie concernait les contraintes émises par le client dont la principale exigence est de réaliser un composant à moindre coût comparé à ceux proposé par le Canada et la Russie (seuls fabricants de machine à temps de vol pour trampoline). Ensuite il fallait un produit transportable, qui puisse s’installer rapidement sur un trampoline et changer de trampoline à tout moment. De plus, le client a demandé à avoir un récapitulatif des différents chronométrages sur Excel à la fin de chaque saut. 

Après l’énoncé des contraintes, nous avons pu proposer quelques fonctionnalités supplémentaires comme l’ajout d’un écran sur le produit pour pouvoir voir pendant les sauts de l’athlète le chronomètre et la réalisation d’un site web dans lequel il y aurait les derniers enregistrements des sauts effectués.

\newpage

%Organisation du groupe
\section{Organisation du groupe}
\subsection{Présentation du groupe}
\paragraph{Alexandre de Lingua de Saint Blanquat :}
Étudiant à l'UPSSITECH, j'ai passé mon baccalauréat au Cambodge avant de faire un IUT GEII. J'ai été accepté à l'UPSSITECH en Système Robotique et Interactif (SRI). J'ai été le chef de projet et j'ai contribué à toutes les parties. J'ai fait la page WEB, la gestion du Git et du GitHub, les tests pour le chronomètre et le programme principal. J'ai aussi proposé au groupe les différents composants et assemblé le prototype. J'étais la personne sur qui les membres du groupe pouvaient compter quand ils étaient en difficultés.

\paragraph{Mohamed EL MOURABIT:}
Étudiant en M1 Système Robotique et Interactifs, je suis issue d'une formation en GEII (Génie électrique et informatique industrielle). J’ai ensuite intégré l’UPSSITECH en L3 en SRI (Système Robotique et Interactifs). Sur ce projet , j'ai principalement travaillé sur la partie programmation. 

\paragraph{Hugo DIRINGER :}
Etudiant en M1 Système Robotique et Interactifs, j’ai débuté pour un DUT GEII (Génie électrique et informatique industrielle). J’ai continué mes étude en intégrant le cursus SRI (Système Robotique et Interactifs) de l’UPSSITECH. Durant ce projet, j’ai eu à m'occuper de plusieurs taches incluant la création et l'utilisation du réseau, l'intégration des différentes partie du projet et la modélisation des boîtier.

\paragraph{Hippolyte Catteau :}
Étudiant en M1 Système Robotique et Interactifs, j'ai d'abord étudié pour obtenir mon DUT GEII (Génie électrique et informatique industrielle). J’ai ensuite intégré l’UPSSITECH en L3 en SRI (Système Robotique et Interactifs). Durant le projet, j'ai été la personne qui était toujours en contact avec le client et qui s'occuper de gérer les différents rendez-vous avec ce dernier. J'ai aidé sur la partie programmation et j'ai réaliser le cahier des charges et le manuel d'utilisation du produit.

\paragraph{François Mahé :}
Étudiant en M1 Système Robotique et Interactifs, j’ai commencé ma formation par une license en Sciences Fondamentales Appliquées suivant une orientation vers l’informatique. J’ai par la suite intégré le cursus de l’UPSSITECH en L3. Durant ce projet j’ai principalement travaillé sur le prototypage du boîtier et l’étude des solutions déjà existantes.

\subsection{Outils utilisés}
Durant la durée du projet nous avons utilisés plusieurs outils :

\paragraph{Outils de gestion de projet :}
\begin{itemize}
    \item Messenger : Communication interne du groupe
    \item Git : gestion du code
    \item GitHub : gestion des différentes tâches du groupes
\end{itemize}

\begin{center}
  \makebox[\textwidth]{\includegraphics[width=13cm]{photoHippo/gestionDeProjetTER}}
  
  Diagramme d'organisation du groupe sur le GitHub
\end{center}

\paragraph{Outils techniques :}
\begin{itemize}
    \item Visual Studio Code : Editeur de texte
    \item Platform IO : Compiler le projet et le téléverser
    \item OpenSCAD et Auto Desk Fusion 360 pour la modélisation 3D des boitiers
    \item Adobe After Effects pour le montage de la vidéo
\end{itemize}

\newpage

%Phase de conceptualisation
\section{Phase de conceptualisation}
\subsection{Produit imaginé}
Pour répondre à tous les besoins de la cliente, nous avons imaginé un produit simple et facile à utiliser. Cas d'utilisation :
\begin{itemize}
    \item Installation simple par branchement des capteurs, puis les aimanter sous le trampoline.
    \item Appuyer sur le bouton du boitier principal pendant un saut pour lancer les mesures.
    \item Vérifier les mesures sur l'écran.
    \item Après dix sauts, le score s'affiche sur l'écran
    \item Une fois les différentes mesures faites, appuyer sur le bouton Wifi, puis se connecter au wifi et accéder au site internet.
    \item Consulter les mesures en détail sur le site internet, télécharger les données sous format Microsoft Excel.
    \item Appuyer à nouveau sur le bouton Wifi pour refaire des mesures de temps.
\end{itemize}

%Les composants :
\subsection{Choix des composants}
Les déférents composants ont été choisis en fonction de leurs couts, de leurs disponibilités et leurs facilités d'utilisation.
\paragraph{Capteur :}
Le composant le plus onéreux. Il est crucial pour la fiabilité des mesures. Nous avons choisi des capteurs laser. Le projet ne partant de rien, nous avons préféré prendre des capteurs de premiers prix. C'est un capteur de type NPN, ayant une distance maximum de 20 mètres et fonctionnant avec des tensions allant de 6 à 20V.

\paragraph{Microcontrolleur :}
Il fallait une cible très accessible avec un module Wifi. Le NodeMCU correspond parfaitement à nos besoins : il possède un module Wifi, fonctionne avec les bibliothèques Arduino, possède un mémoire flash de quatre mégaoctets et à une horloge de 80Mz permettant une lecture rapide des entrées.

\paragraph{Ecran :}
Nous avons choisi un  écran Oled, fonctionnant uniquement en I2C, pour la facilité d'utilisation et la lisibilité de l'image en pleine lumière. Il est petit et peut facilement être remplacé pour un plus grand modèle si besoin est.

\paragraph{Alimentation :}
Nous avons préféré utiliser des modules peu puissants. c'est pour cela que nous avons pris deux transformateurs 5V, 2A. Ces tensions sont suffisamment faibles pour ne pas être dangereuses en cas d'électrocution ou en cas de cours circuit. Ils sont ce dont nous avons besoin pour tester, mais devront être augmentés pour la partie émetteur laser, car une tentions de minimum 6V est demandé. Nous avons fait fonctionner le capteur sous 9V et sous 5V. 5V est faible, mais suffisant, car la distance sous le trampoline est courte. 9V est optimal, mais le rayon laser est plus puissant et du coup plus dangereux.

\paragraph{Connectique :}
Nous avons pris une connectique très accessible. Pour la partie alimentation, nous avons pris des connecteurs $5.5mm * 2.1mm$. Pour la partie récepteur, il nous fallait des connecteurs 3 pins. Nous avons choisi de prendre des connecteurs Jack $6.3mm$.

\subsection{Architecture logicielle}
La partie logiciel nous permet de mettre en relation tous les composants. La fonction première du produit étant de chronométrer, nous comptions utiliser l'horloge interne du microcontrôleur pour faire cela. Comme dit précédemment, l'horloge interne tourne à une vitesse de 80MHz (soit à 12,5 nanosecondes d'intervalle). Pour ne pas avoir de perte de précision dans les données, nous avons choisi de stocker toutes les mesures en microsecondes puisque c'est ce qui est utilisé en interne par la bibliothèque Arduino. La précision recherchée étant la milliseconde, cela nous convenait. Les valeurs sont forcément positives, nous avons donc utilisé des nombres non signés. Le microcontrôleur étant construit sur une architecture à 32 bits, stocker les mesures sur des nombres 32bis était le plus pertinent, sachant que la valeur maximum d'un temps en microsecondes non signé sur 32 bits revient à environ 71 minutes. Dans le cas où une valeur dépasserait la maximale, le compteur recommencerait à zéro. 

Ci-dessous le diagramme de classe alors imaginé pour la partie logiciel.

\begin{center}
  \makebox[\textwidth]{\includegraphics[width=15cm]{photoAlex/diagrammeDeClasse_Ancien.png}}
  
  Diagramme de classe initialement prévu pour la partie logiciel.
\end{center}

\newpage

%Phase de conception
\section{Phase de conception}

%Conception Logiciel
\subsection{Conception Logiciel}
\subsubsection{Chronomètre}
La première version du chronomètre avait comme objectif de traiter un signal non bruité (voir figure 1). On considère donc un signal sans perturbation où l’on mesure la durée de l’état haut, qui correspond à la position de l’athlète dans l’air. On lance le chronomètre à chaque front montant et on l’arrête à chaque front descendant.

\begin{center}
\makebox[\textwidth]{\includegraphics[width=17cm]{photoMohamed/diagramme1ereversion.png}}

Figure 1 – Signal simple

\end{center}

On a ensuite réalisé une autre version du chronomètre qui prenait en compte les rebonds. Notre objectif était d’éliminer les perturbations liées aux rebonds, causés par les vibrations du trampoline. On a ensuite identifié deux types de rebonds. Une première catégorie de rebonds dus aux vibrations causées par la personne présente sur la toile et une deuxième catégorie de rebonds liés aux lasers qui passent entre le maillage de la toile du trampoline. (Voir figure 2 et 3)

\begin{center}
\makebox[\textwidth]{\includegraphics[width=17cm]{photoMohamed/diagramme2.png}}
  
Figure 2 – Signal perturbé par le décollage d’un individu
\end{center}

\begin{center}
\makebox[\textwidth]{\includegraphics[width=17cm]{photoMohamed/diagramme3.png}}
  
Figure 3 – Signal perturbé par l’atterrissage d’un individu 
\end{center}

Notre solution à ce problème est l’implémentation d’un seuil antirebonds qui prenait en compte le temps minimal de chaque état, et qui refuse n’importe quelle mesure en dessous de ce seuil. Cette solution traite un signal (voir ci-dessus), où l’on élimine chaque état haut qui a une durée inférieure au seuil établi. Après avoir réalisé quelques tests, nous avons choisi un seuil de 300ms. Ce seuil prend en compte qu’une personne reste en l’air au minimum 500 ms. Le seuil est donc suffisamment grand pour éliminer tout type de perturbation, mais suffisamment petit pour prendre en compte les plus petits sauts.

Cependant notre objectif final est d’implémenter un chronomètre qui puisse gérer des perturbations liées aux interférences(perturbation liée à la coupure du faisceaux laser par des éléments externes, indépendamment des fronts). Sachant que lors de nos tests dans le gymnase nous n'avions pas constaté d’interférences, il fallait quand même implémenter cette fonctionnalité car c'est un cas qui pourrait entraîner des erreurs de mesures. Ce seuil prend en compte l’état bas du signal. Si celui-ci est inférieur au seuil, on ne le prend pas en compte et l'on revient au dernier front montant, ce qui nous permet d’éliminer le bruit et d'avoir une mesure plus robuste. (Voir figure 4 ci-dessous).

\begin{center}
    \makebox[\textwidth]{\includegraphics[width=17cm]{photoMohamed/diagramme4.png}}
      
    Figure 4 – Signal bruité par des interférences
\end{center}

Notre programme Chronomètre (ou modeMesure) arrive à transformer un signal bruité, lié à des interférences ou des rebonds et simule un signal sans bruit ou l’on mesure l’état haut à chaque front montant. (Voir exemple figure 5) 

\begin{center}
    \makebox[\textwidth]{\includegraphics[width=19cm]{photoMohamed/diagramme5.png}}
      
    Figure 5 – Signal avant et après traitement
\end{center}

\subsubsection{Tableau des mesures}
\begin{center}
    \makebox[\textwidth]{\includegraphics[width=13cm]{photoMohamed/figure6.png}}
      
    Figure 6 – Diagramme UML TableauDesMesures
\end{center}

Cette classe a comme fonction d'écrire et lire dans la mémoire Flash. Nous avons créé cette classe qui utilise comme variable une matrice de 20 tableaux de 10 mesures en entier non signées de 32 bits. L’idée est d’utiliser ce tableau comme intermédiaire pour écrire ou lire dans la mémoire. Donc nos mesures sont enregistrées dans ce tableau, de façon à lire la dernière mesure que l’on a stockée. Ceci est appelé LIFO (Last in, First out). À chaque nouvelle acquisition, les 10 mesures sont stockées dans ce tableau à l'emplacement du curseur (index\_write). On a ensuite développé des fonctions qui permettent d’écrire toute la matrice dans la mémoire, grâce à la librairie Arduino qui nous fournissait ces fonctionnalités. L'écriture se fait par paquet de 8 bits. Nous avons créé notre propre fonction, qui décomposait nos entiers non signés de 32 bits en 4 morceaux d’entiers non signés de 8 bits que nous pouvions ensuite écrire ou lire dans la mémoire Flash.

\begin{center}
    \makebox[\textwidth]{\includegraphics[width=15cm]{photoMohamed/figure7.png}}
      
    Figure 7 – Décomposition de uint32 en uint8
\end{center}

\subsubsection{Site internet}
Le site internet doit afficher les mesures de temps de vol. Pour cela nous avons décidé d'afficher deux formats des données en parallèle : un tableau avec les mesures de chacun des sauts et un graphique prenant le numéro du saut en abscisse et le temps de vol en ordonnée.

Le site internet a été codé en HTML, CSS et JavaScript. En effet, toute la partie affichage de la page est faite du côté client, au travers du navigateur. Le serveur ne fait qu'envoyer le code de la page. Il fallait prendre en compte qu'il n'y a pas d'accès à internet sur le réseau local mis en place par la machine à temps de vol. Cela implique que l'inclusion d'une bibliothèque doit forcément se faire en ajoutant le code source de la bibliothèque au code. Il fallait aussi prendre en compte que la taille de la mémoire flash est de quatre mégaoctets. Malgré ces contraintes nous devions afficher un graphique représentant les mesures. Dans un premier temps, nous avons essayé de trouver une librairie JavaScript suffisamment légère et utilisable à des fins commerciales pour ne pas avoir de restriction en cas d'évolution du projet. Malheureusement, nous n'avons pas réussi à trouver et c'est pour cela qu'il a fallu coder nous-mêmes une librairie d'affichage de graphiques.

Le microcontrôleur a peu de RAM et la gestion de chaîne de caractère très grande est problématique. Le code a donc été fait pour que le JavaScript génère lui-même le code HTML et CSS. Cela permet de déplacer la partie génération du code sur le navigateur du client. Donc nous injectons directement les mesures dans le JavaScript. C'est ce qui est fait dans la classe HTMLgenerator : elle ajoute les mesures en dure dans le code HTML. Pour cela le tableau de mesures est converti en une chaîne de caractère correspondant à la déclaration du tableau en Javastring.

Les mesures sont ensuite traitées par le site pour être affichées. La cliente nous a explicitement demandé de pouvoir récupérer les mesures sur Microsoft Excel. Cette fonctionnalité a demandé beaucoup de recherches, car il ne fallait pas utiliser une bibliothèque trop grande. Finalement une solution très simple a été trouvée en appelant directement l'application Excel depuis le navigateur plutôt que de générer un fichier de sauvegarde standard. Cette solution ne fonctionne que si Excel est installé, mais après plusieurs tests, plusieurs autres applications de tableur sont compatibles avec cette commande, notamment sur mobile. Si vous êtes intéressé, vous pouvez vous reporter à la fonction « fnExcelReport() » dans le document data/testPageWeb.html. Cette fonction est disponible depuis un bouton « Export to Excel » en haut de la page.

Le dernier cas traité est le cas ou il n'y a pas de mesures. Dans cette situation une page blanche avec le message « Il n'y a pas de mesures » est envoyée.

\begin{center}
  \includegraphics[width=10cm]{photoAlex/siteWEB}
  
  Exemple de deux mesures sur le site internet avec le bouton d'exportation.
\end{center}

\subsubsection{Serveur}
Le serveur a pour but d'afficher le site internet. Il a été codé en C++ à l'aide de plusieurs bibliothèques déjà prévues à cet effet, pour notre "nodeMCU V3" précisément. Le serveur a pour lui 2 bibliothèques : 
\paragraph{ESP8266WebServer.h :}
Elle nous a permis de démarrer et de configurer notre serveur. Incluant des fonctions permettant de définir les modes, les adresses IP, les masques et les différents identifiants et mots de passe, cette bibliothèque nous a permis de générer le serveur en un minimum de lignes de code.
Elle incluait, en plus, des fonctions nous permettant d'afficher nos pages WEB et de générer plusieurs pages en fonction de l'URL rentrée (commençant par notre IP). N'ayant qu'une seule page à afficher, nous l'affichons pour importer l'URL tant qu'il commence par l'IP voulu.
\paragraph{DNSServer.h :}
Celle-là nous a permis d'utiliser un serveur DNS et de rediriger une adresse web vers une autre. Grâce à elle, nous avons pu rediriger l'adresse "www.tempsdevol.com" vers une URL composée de notre adresse IP et cela en seulement 3 lignes.
\paragraph{}
Le serveur a donc été créé et géré grâce à ses 2 bibliothèques. Il a fallu faire beaucoup de recherches pour apprendre à les connaître et utiliser leurs fonctions, mais le serveur a été fait relativement vite par la suite.

\subsubsection{Test}
Chaque classe de notre programme a été faite séparément en utilisant des données simulées et le diagramme de classe fait au début du projet. Les lasers mettant longtemps à arriver, nous avons commencé par tester nos programmes avec de simples boutons. Nous avons d'abord simulé nos lasers avec ses boutons pour tester notre chronomètre puis l'avons ensuite testé avec les vrais lasers. Nous avons alors remarqué quelques rebonds dans les signaux renvoyé capteurs lasers qui nous affichaient de fausses mesures. Il a alors fallu simuler les rebonds informatiquement et modifier le chronomètre pour qu'il ne soit plus affecte par eux. Une fois la programmation de nos parties respectives terminée, nous avons commencé à les intégrer dans un ordre prédéfini afin de diminuer les erreurs qui risquaient d’apparaître par la suite. L’ordre des intégrations était : Réseau$\,\to\,$PageWeb$\,\to\,$chronomètre/temps de vol$\,\to\,$écran$\,\to\,$programme principal. Le but de chacune de ces intégrations était :
\paragraph{Réseau :}
Nous avons commencé par vérifier que notre serveur est capable d’afficher une page internet correctement.
\paragraph{PageWeb :}
Le but de cette première intégration était alors de générer des pages web différentes avec des données simulées, puis de vérifier que le réseau actualise bien la page web affichée. 
\paragraph{Chronomètre/temps de vol :}
Le but de cette partie étant de mesurer une donnée extérieure, nous avons fixé les lasers utilisés de façon à simuler l’utilisation du chronomètre sur le trampoline. Il fallait alors retourner les mesures de chronomètre pour générer la nouvelle page web puis l’afficher sur le réseau.
\paragraph{Écran :}
Il fallait alors retourner les mesures de chaque saut et le temps total vers l’écran pour nous permettre d’utiliser la machine à temps de vol sans avoir un appareil avec une connexion au réseau.
\paragraph{Programme principal :}
Pour finir, il fallait alors intégrer les 3 lasers dans le programme (nous n’en utilisions qu’un seul jusqu’à maintenant) et de rajouter une LED indiquant si oui ou non les lasers sont coupés (ou s’ils ne sont pas bien mis en place)

%programme principale
\subsubsection{Programme principal}
Le programme principal est assez simple puisque toutes les fonctionnalités les plus importantes ont été déportées dans les classes. Le programme principal gère la lecture des entrées et l'allumage de la LED au travers d'interruptions et à l'aide des différentes classes créées. Le diagramme qui suit est le diagramme de classe final. Il est plus grand que le diagramme initial et montre bien l'évolution du projet entre la conceptualisation et la conception.

\begin{center}
  \makebox[\textwidth]{\includegraphics[width=19cm]{photoAlex/diagrammeDeClasse.png}}
  
  Diagramme de classe final
\end{center}

%Réseau
\subsection{Réseau}
Le composant que nous avons choisi d’utiliser, le « nodeMCU V3 », a l’avantage d’intégrer directement un module WIFI et le réseau a pu être intégré grâce à lui. Ce module reste tout de même une partie à part du composant. Pour l’utiliser, il fallait tout d’abord flasher la carte pour paramétrer le module Wifi et y assigner toutes les données dont il fallait disposer pour l’utiliser, telles que les adresses MAC.
\begin{center}
    \includegraphics[scale=0.4]{photoHugo/image001}
    
    Flash du module WIFI
\end{center}
Une fois fait, les paramètres étaient gardés en mémoire et il s’agissait alors de définir les paramètres de notre serveur pour utiliser le réseau comme nous le voulions : le mode, l’adresse IP, le masque, l’identifiant et le mot de passe. Le module WIFI a à disposition 2 modes : le mode point d’accès (AP) et le mode relais (STA) permettant de relayer un réseau existant. Les modes étant initialement tout les deux activés, nous commençons donc par forcer l’activation du mode AP uniquement. L’adresse IP a été fixée à 192.168.4.1 et le masque à 255.255.255.0 permettant ainsi à 254 hôtes de se connecter au réseau (en supposant que notre composant le supporte). L’identifiant du réseau est « tempsDeVol » afin d’être facilement reconnaissable, nous ne divulguerons cependant pas le mot de passe qui a seulement été donné à notre cliente. N’ayant qu’une page internet à afficher, celle-ci s’affiche automatiquement lorsque l’on tape l’adresse IP dans la barre de recherche.
\begin{center}
    \makebox[\textwidth]{
        \includegraphics[width=0.6\textwidth]{photoHugo/Capture3}
        \includegraphics[width=0.6\textwidth]{photoHugo/Capture2}
        }
    
    Connection à la page web
\end{center}

\normalsize
Pour faciliter la recherche, nous utilisons un serveur DNS pour rediriger une autre adresse internet simple à retenir vers notre adresse IP. Nous avons donc choisi de rediriger l’adresse « www.tempsdevol.com » pour y retrouver nos mesures.

\begin{center}
    \includegraphics[width=0.9\textwidth]{photoHugo/Capture1}

    Redirection de l'adresse "www.tempsdevol.com"
\end{center}

\normalsize
%Conception des boîtiers et supports
\subsection{Boîtiers et supports}
\subsubsection{Supports récepteurs}
Au vu des avancées de l’équipe sur les programmes gérant la page internet, le réseau et la mesure du temps de vol, il a été rapidement nécessaire de pouvoir mettre le programme de mesure du temps de vol à l’épreuve des conditions réelles. Nous ne savions alors pas vraiment comment le trampoline allait se comporter durant un exercice en condition de compétition, mais nous savions qu’il pourrait avoir un impact sur les performances du programme de mesure. Il était nécessaire d’imaginer les scénarios pouvant poser des problèmes : que se passe t-il si on ne finit pas la série de dix sauts ? Que se passe t-il si un athlète reste sur la toile ? L’effet de tous ces cas non conformes à une utilisation standard ont nécessité un prototype pour pouvoir être finement appréhendés. De plus, nous devions, à terme, mettre au point un dispositif robuste aux oscillations mécaniques, autrement dit, aux vibrations, et nous voulions pouvoir expérimenter des prototypes au plus vite.

Le premier prototype est une simple boîte, permettant d’y loger un pointeur LASER, ou un récepteur. Cette boîte était faite pour être assemblée grâce à des accessoires facilement trouvables en boutique de bricolage : un jeu de 4 vis et un autre de 4 aimants. Ce prototype nous a principalement permis de comprendre qu’une aimantation assez forte contre le cadre horizontal du trampoline serait suffisante pour avoir un système convenablement stable. Il a également permis à l’équipe développant le programme de mesure de comprendre , qu’il était obligatoire de considérer des problèmes dit de rebond; phénomène d'oscillations non contrôlés pouvant être observés sur un signal électrique, parfois dus au dispositif générant le signal, ici, les récepteurs LASER, et parfois dû à l’équipement sujet à la mesure, ici notamment, l’aspect maillé de la toile du trampoline.

\begin{center}
  \makebox[\textwidth]{\includegraphics[width=8cm]{photoFrancois/image1}}
  
  Prototype initial
\end{center}

Il a permis aussi de voir émerger l’idée de positionner les capteurs et les récepteurs sur la largeur du trampoline, et non plus sur la longueur, afin de les rapprocher, la distance ayant un impact direct sur l’observation de l’amplitude de l'oscillation d’un point projeté.

\begin{center}
  \makebox[\textwidth]{\includegraphics[width=15cm]{photoFrancois/image2}}
  
  Trajectoire du laser
\end{center}

Le prototype final prend en compte tous les aspects déjà évoqués, il permet principalement d’être conforme au nouveau support pour l'émetteur, qui se devait d’être plus facilement réglable. Le support de l'émetteur sera décrit plus en avant dans ce document.

\begin{center}
  \makebox[\textwidth]{\includegraphics[width=8cm]{photoFrancois/image3}}
  
  Prototype final du récepteur
\end{center}

\subsubsection{Supports laser}
Les supports des lasers (émetteur et récepteur) sont tout deux partis du boîtier les lasers récepteurs que nous avons graduellement améliorés pour corriger deux problèmes : la fixation du laser contre le trampoline et la modification de la hauteur du laser pour qu’il atteigne le récepteur. Toujours à l’aide d’imprimante 3D, nous sommes partis du boîtier de base et avons pensé à plusieurs prototypes :

\paragraph{Prototype n°1 :}
Le but était de lever ou baisser l’arrière du laser pour en modifier la hauteur. Ce n'était pas assez fiable et il n'y avait pas assez de marge de manœuvre.
\begin{center}
    \includegraphics{photoHugo/image005}
    
    Prototype n°1 
\end{center}

\paragraph{Prototype n°2 :}
Un système mécanique entièrement imprimé en 3D : il se serait vite abîmé et n’aurait pas eu une bonne durée de vie.
\begin{center}
    \includegraphics{photoHugo/image007}
    \includegraphics{photoHugo/image009}
    
    Prototype n°2
\end{center}

\paragraph{Prototype n°3 :}
Une meilleure mise en place des aimants et une modification de la hauteur du laser fait par de vrais vis et écrous. Une meilleure durée de vie, mais l'on remarque que les aimants ne sont pas assez puissants lorsque des adultes sautent.
\begin{center}
    \includegraphics{photoHugo/image011}
    
    Prototype n°3
\end{center}

\paragraph{Prototype n°4 :}
Ajout de nouveaux aimants plus puissants sur les côtés du boîtier. Le système est fonctionnel, mais la forme de boite n’étant plus utile pour poser les aimants sur le couvercle, on revoit toute la structure.
\begin{center}
    \includegraphics{photoHugo/image013}
    
    Prototype n°4
\end{center}

\paragraph{Prototype n°5 :}
Nous partons alors sur un système beaucoup plus minimaliste ou l’on utilise une vis pour modifier l’inclinaison du laser et l'on visse un écrou pour fixer la position du laser : le système est utilisable, mais il y a un petit jeu sur les aimants et l’on doit forcer sur l’écrou (même avec une aide) pour que le laser ne bouge plus.
\begin{center}
    \includegraphics{photoHugo/image015.png}
    
    Prototype n°5
\end{center}

\paragraph{Prototype n°6 :}
Nous réglons les 2 derniers problèmes, une vis de serrage rapide est ajoutée pour faciliter la fixation du laser et les aimants sont enlevés de leurs supports de base et fixés manuellement : le prototype est entièrement fonctionnel. 
\begin{center}
    \includegraphics[width=0.4\textwidth]{photoHugo/image017}
    
    Prototype n°6
\end{center}

Nous arrivons finalement à un prototype entièrement fonctionnel et économe en quantité de plastique et de temps d’impression. Après avoir fini de prototyper la fixation de nos lasers émetteurs, nous avons refait la fixation du laser récepteur de la même façon. Même si le prototype actuel est fonctionnel, nous pourrions encore l'améliorer à l'avenir. 
\begin{center}
    \includegraphics[width=0.4\textwidth]{photoHugo/image019}
    \includegraphics[width=0.4\textwidth]{photoHugo/image021}
    
    Prototype final (laser émetteur et récepteur)
\end{center}

\subsubsection{Boîtiers}
En plus de la fixation des lasers, nous avons imprimé plusieurs boîtiers afin d’y fixer les connectiques de nos lasers, les alimentations et de protéger l’utilisateur des soudures. Nous avons alors eu deux boîtiers à imprimer :
\paragraph{Le boîtier d’alimentation}
Il a pour but d’alimenter les 3 lasers émetteurs présents du côté opposé à celui où va se trouver l’utilisateur de la machine à temps de vol. Il prend en entrée 1 jack d’alimentation (le câble étant fourni) et en sortie les 3 jacks des 3 lasers émetteurs.
\begin{center}
    \includegraphics[width=0.4\textwidth]{photoHugo/image023}
    \includegraphics[width=0.4\textwidth]{photoHugo/image025}
    
    Boîtier d'alimentation
\end{center}
\paragraph{Le boîtier de contrôle}
Il s’agit du boîtier que l’utilisateur va avoir en main. Il inclut les 3 jacks des 3 lasers récepteurs, 1 jack pour l’alimentation et un accès aux interfaces présents sur le composant (écran, bouton et LED). 
\begin{center}
    \includegraphics[width=0.4\textwidth]{photoHugo/image027}
    \includegraphics[width=0.4\textwidth]{photoHugo/image029}
    
    Boîtier de contrôle
\end{center}

%Câblage
\subsection{Câblage}
Le schéma qui suit est le câblage qui a été fait sur le prototype. La grille représente la carte PCB de prototypage qui est prépercée suivant cette grille. Les ronds blancs représentent les séparations entre les composants et les carrés sont les pins du microcontrolleur, de l'écran, ou des capteurs.
\begin{center}
    \makebox[\textwidth]{\begin{circuitikz}[european,scale=0.85]

%generation de la grille
\newcounter{ct}

\forloop{ct}{0}{\value{ct} < 15}
{
  \draw[dashed] (-0.5 + \value{ct}, -0.5) -> (-0.5 + \value{ct}, -0.5 + 20);
}

\forloop{ct}{0}{\value{ct} < 21}
{
  \draw[dashed] (-0.5, -0.5 + \value{ct}) -- (-0.5 + 14,-0.5 + \value{ct});
}

\forloop{ct}{1}{\value{ct} < 15}
{
  \node at (\value{ct}-1,20){\arabic{ct}};
}

\forloop{ct}{1}{\value{ct} < 21}
{
  \node at (14,21 - \value{ct}-1){\Alph{ct}};
}

%nodeMCU
\node[draw] (D0) at (1,15) {$D0$};
\node[draw] (D1) at (1,14) {$D1$};
\node[draw] (D2) at (1,13) {$D2$};
\node[draw] (D3) at (1,12) {$D3$};
\node[draw] (D4) at (1,11) {$D4$};
\node[draw] (3V1) at (1,10) {$3V$};
\node[draw] (G1) at (1,9) {$G$};
\node[draw] (D5) at (1,8) {$D5$};
\node[draw] (D6) at (1,7) {$D6$};
\node[draw] (D7) at (1,6) {$D7$};
\node[draw] (D8) at (1,5) {$D8$};
\node[draw] (RX) at (1,4) {$RX$};
\node[draw] (TX) at (1,3) {$TX$};
\node[draw] (G2) at (1,2) {$G$};
\node[draw] (3V2) at (1,1) {$3V$};

\node[draw] (A0) at (12,15){$A0$};
\node[draw] (G3) at (12,14) {$G$};
\node[draw] (VU) at (12,13) {$VU$};
\node[draw] (S3) at (12,12) {$S3$};
\node[draw] (S2) at (12,11) {$S2$};
\node[draw] (S1) at (12,10) {$S1$};
\node[draw] (SC) at (12,9) {$SC$};
\node[draw] (S0) at (12,8) {$S0$};
\node[draw] (SK) at (12,7) {$SK$};
\node[draw] (G4) at (12,6) {$G$};
\node[draw] (3v3) at (12,5) {$3V$};
\node[draw] (EN) at (12,4) {$EN$};
\node[draw, scale=0.8] (RST) at (12,3) {$RST$};
\node[draw] (G5) at (12,2) {$G$};
\node[draw, scale=0.8] (VIN) at (12,1) {$VIN$};

%PIN ecran
\node[draw, scale=0.8] (VCC) at (5,18) {$VCC$};
\node[draw, scale=0.75] (GND) at (6,18) {$GND$};
\node[draw, scale=0.8] (SCL) at (7,18) {$SCL$};
\node[draw, scale=0.8] (SDA) at (8,18) {$SDA$};

%LED
\draw (D0.north) -- (1,19) to[short,-o] (13,19) to[short,-o, empty led] (13,18) to[short,-o] (13,17) to[R={\parbox{1cm}{\SI{100}{\ohm}}}] (13,14) to[short,o-] (G3.east);

%Bouton (bas)
\draw (3V2.east) to[short,-o] (2,1) to[R={\parbox{1cm}{\SI{1000}{\ohm}}}] (6,1) to[short,o-] (6,2);

%Bouton (haut)
\draw[brown] (6,7) to[short] (6,8) to[short] (D5.east);
\draw (4,7) to[short] (4,9) to[short] (G1.east);

%Bouton
\draw
    (6,2) to[short,o-o] (6,7)
    (4,4.5) to[push button] (6,4.5)
    (4,2) to[short,o-o] (4,7)
;
\node[text width=3cm] at (6,5.8){Bouton};

%ecran
\draw[red] (VU.west) to[short] (9,13) to[short] (9,16) to[short] (5,16) to[short] (VCC.south);
\draw (G3.west) to[short] (10,14) to[short] (10,17) to[short] (6,17) to[short] (GND.south);
\draw[blue] (D1.east) to[short] (8,14) to[short] (SDA.south);
\draw[green] (D2.east) to[short] (7,13) to[short] (SCL.south);
\node[text width=3cm] at (6.2,18.8){Ecran};

%alim
\draw[red] (12, -1) to[short, o-] (12,0) to[short] (13,0) to[short] (13,13) to[short] (VU.east);
\draw (10, -1) to[short, o-] (10,2) to[short] (G5.west);

\draw (10, -1) to[V=5V] (12, -1);

%capteur 1
\node[draw, scale=0.8] (C1O) at (-2,12){$OUT$};
\draw (C1O.east) to[R={\parbox{1cm}{\SI{1000}{\ohm}}}] (0,12) to[short, o-] (D3.west);
\node[draw, scale=0.8] (C1G) at (-3,12) {$GND$};
\node[draw, scale=0.8] (C1V) at (-4,12) {$VCC$};
\node[text width=3cm] at (-4.5,12.5){Capteur 1};

%capteur 2
\node[draw, scale=0.8] (C2O) at (-2,7){$OUT$};
\draw (C2O.east) to[R={\parbox{1cm}{\SI{1000}{\ohm}}}] (0,7) to[short, o-] (D6.west);
\node[draw, scale=0.8] (C2G) at (-3,7) {$GND$};
\node[draw, scale=0.8] (C2V) at (-4,7) {$VCC$};
\node[text width=3cm] at (-4.5,7.5){Capteur 2};

%capteur 3
\node[draw, scale=0.8] (C3O) at (-2,6){$OUT$};
\draw (C3O.east) to[R={\parbox{1cm}{\SI{1000}{\ohm}}}] (0,6) to[short, o-] (D7.west);
\node[draw, scale=0.8] (C3G) at (-3,6) {$GND$};
\node[draw, scale=0.8] (C3V) at (-4,6) {$VCC$};
\node[text width=3cm] at (-4.5,6.5){Capteur 3};

\draw 
    (C1G.south) -- (C2G.north)
    (C2G.south) -- (C3G.north) 
    (C3G.south) -- (-3,-1) -- (10, -1)
    ;
    
\draw[red]
    (C1V.south) -- (C2V.north)
    (C2V.south) -- (C3V.north) 
    (C3V.south) -- (-4,-2) -- (12, -2) -- (12, -1)
    ;

\end{circuitikz}
}
\end{center}

\subsection{La conception de la video}
La vidéo avait pour but de présenter rapidement le contexte de notre travail et ce que nous avions réalisé. Nous voulions un plan avec la cliente expliquant son idée et ce pour quoi elle avait fait appel à l’Upssitech. Le montage des différents plans s’est fait à l’aide du logiciel AfterEffect, un logiciel de montage semi-professionnel très souvent utilisé par les monteurs intervenant sur des vidéos diffusées sur internet. L’idée était de faire se succéder différents plans structurants sans trop détailler le cheminement du travail que nous avions réalisé, jusqu'à la démonstration finale, présentant sous différent points de vus synchronisés, le comportement du produit en situation réelle. La musique choisie est libre de droit, Up \& Away de JPB.

\begin{center}
  \makebox[\textwidth]{\includegraphics[width=18cm]{photoFrancois/image4}}
  $www.youtube.com/watch?v=wKbeXyNoAr4$
\end{center}

\newpage

%Prototype final
\section{Prototype final}
Le prototype final a été testé en condition réelle. Pour faciliter le branchement des capteurs au boitier principal, nous avons utilisé des prises Jack 6.3mm. Un défaut de conception fait que le branchement de ces prises provoque un court circuit très rapide, mais suffisant pour faire redémarre le microcontrôleur. Ce court circuit n'est pas dangereux, car le module d'alimentation est peu puisant, mais il reste à éviter au maximum. C'est pour cela qu'il est recommandé de brancher les capteurs avant l'alimentation au boitier principale. Le test en condition réel nous a permis de tester différents aspects :
\subsection{Installation}
Le but du projet aussi était de rendre le projet accessible à des personnes n'ayant pas de connaissance en électronique. Comme le produit est transportable et peut s'installer sur n'importe quel trampoline nous avons rédiger un manuel d'utilisation avec la partie la plus importante qui est l'installation du produit sur un trampoline. Nous l'avons rédiger sous forme d'étape afin que cette partie sois la plus claire et simple possible. Nous l'avons écrite comme suis:

\begin{itemize}
    \item Brancher les récepteurs lasers aux prises jack du boitier de contrôle.
    \item Mettre la machine sous tension.
    \item Positionner un émetteur laser dans la diagonale du trampoline, positionner les deux autres sur les côtés dans le sens de la largeur du trampoline. Les boitiers sont aimantés et peuvent donc s’attacher au barre métallique du trampoline
    \item Mettre en face les récepteurs lasers. Régler les émetteurs de sorte que le laser arrive bien sur le récepteur laser. A ce moment-là, la lumière derrière le récepteur devrait s’éteindre. 
    \item La machine est prête à l’utilisation
\end{itemize}

\subsection{Mesure du temps}
La mesure du temps est faite pour calculer le temps de vols d'athlète sur un enchainement de dix sauts sur un trampoline. Sachant qu’il faut aussi qu'un membre externe regarde l’athlète pour déclencher le comptage des sauts en appuyant sur le bouton. Le bouton va déclencher ce chronométrage d'une série de 10 mesures en commençant par celle en cours. À chaque fois, le temps de vols de la figure est affiché sur l'écran, ainsi qu'à la fin le score total.
Notre mesure du temps de vols est assez cohérente avec la réalité, avec les tests qu’on a faits sur le microcontrôleur, on arrive à une précision de quelques microsecondes. La seule chose que l’on n’a pas pu tester est la comparaison de nos mesures avec une autre machine de temps de vols, ce qui ne nous permettrait pas de savoir si nos mesures sont pertinentes. 

\subsection{Consultation des données}
La consultation des données est assez réussite. Le score est directement affiché après les sauts sur l'écran et les détails sont sur le site internet. L'ajout de la mémorisation a permis de garder les vingt dernières mesures même si la machine à temps de vol est débranchée. Pour améliorer le produit, il faudrait maintenant ajouter de l'interactivité avec le site internet. Par exemple renommer une mesure, en supprimer une ou en épingler une pour qu'elle reste définitivement. La LED est finalement peu pertinente. Elle peut servir à vérifier que les capteurs fonctionnent bien, mais ne remplace pas la vérification des capteurs individuelle. Il faudrait implémenter un système qui compare les capteurs pour vérifier si l'un d'entre eux est désynchronisé. Néanmoins, pour le temps que nous avions, le résultat reste très satisfaisant.

\subsection{Supports et boîtiers}
Les boîtiers et supports utilisés pour la machine à temps de vol sont fonctionnel mais peuvent encore être améliorer. Nous avons pensé a plusieurs points améliorable à l'avenir tel que le prix et la qualité du produit. Il faut soulever un point important qui l’achat de matériel supplémentaire en plus des impression 3D: des aimants et des vis de serrage rapide. Il serait possible de d'améliorer la qualité du matériel en achetant des produit plus haut de gamme. Il serait aussi possible d'améliorer la qualité des impressions améliorant les dimensions de nos impression pour mieux caler nos aimants (à l'échelle du dixième de mini mètre). Pour finir, nous pourrions réduire la taille du boîtier de contrôle en modifiant la place des composant pour réduire l'espace occupé et rajouter une rondelle auto-bloquante pour stabilisé d'autant plus le laser. Malgré tout ça, nous pouvons être fière d'avoir un produit fonctionnel à l'heure actuel et qui reste en place pendant une durée de plus de 2h sans avoir à re-régler la position des lasers (testé avec des gymnastes).
\newpage

%Bilan
\section{Bilan}
\subsection{Bilan individuel}
\subsubsection*{Alexandre de Lingua de Saint Blanquat}
Ce projet m'a beaucoup appris sur le travail de groupe. Au début, je n'étais pas chef de projet, car nous avions choisi d'avoir une organisation sans hiérarchie, mais le projet stagnait et prenait du retard. J'ai pris alors plusieurs initiatives sur le choix des composants et organisé plusieurs réunions avec le groupe et les autres membres ont adhéré et le projet à commencer à décoller. Le problème que nous n’avions pas de fil conducteur, alors je me suis investi pour guider notre équipe. J'espère que le résultat témoigne de mon investissement.

\subsubsection*{Mohamed El-Mourabit}
Mon appréciation globale sur ce projet est très positive. J'ai trouvé très intéressant de pouvoir travailler sur un projet où l'on voit notre objectif final depuis le début. On a eu une complète autonomie tout au long du projet et on a pu développer avec une totale liberté toutes les parties du produit. Notre cliente était très disponible pour répondre à nos questions, ce qui nous a permis de comprendre ses besoins. On a eu un accès total au gymnase pour pouvoir tester nos différents prototypes, et les athlètes étaient tout à fait disposés à nous fournir de l'aide.

\subsubsection*{Hugo Diringer}
Le projet que nous avons eu à faire durant ce semestre a été très enrichissant. Nous avions défini un cahier des charges avec notre cliente puis nous avons eu la liberté d'utiliser les logiciels, langages de programmation et matériel que nous souhaitions pour atteindre nos objectifs. Le fait de démarrer de rien nous a permis d'acquérir de l'expérience et de pouvoir expérimenter pour la première fois la conduite d'un projet du début à la fin en total liberté. De plus, notre cliente était à notre disposition pour nous donner les informations dont nous avions besoin ainsi que pour avoir un trampoline à disposition pour tester notre prototype. Tout cela nous a fortement aidé lors de la réalisation du projet et le fait de pouvoir venir sur place et faire nos tests dans la bonne ambiance était très sympathique. 

\subsubsection*{Hippolyte Catteau}
Mon retour sur le projet est très positifs et surtout très instructif. Tout d'abord nous avons été très bien accueilli par le client et au gymnase. En suivant les horaires donnés nous avons toujours eu accès à un trampoline libre durant notre créneau au gymnase pour pouvoir tester notre produit. Aussi, le projet s'est bien déroulé au sein du groupe grâce a une bonne entente et une confiance mutuelle.
Ensuite d'un point de vue technique, ce qui m'a le plus apporté est le fait d'avoir réaliser un projet de A à Z dans le format "projet d'entreprise". Le client est venu nous voir avec un produit, nous avons établi un cahier des charges ensemble et nous sommes passés par toute les étapes de réalisation d'un projet. Cela nous a permis de mettre en pratique toute nos connaissances scolaire.
Pour finir, nous avons réussir a rendre un produit fonctionnel et respectant le cahier des charges et il n'y a rien de plus satisfaisant d'autant plus que le client avait l'air d'apprécier notre travail.

\subsubsection*{Mahe François}
Ce projet fut très intéressant. C’est la première fois pour moi que j’ai pu être en contact direct avec un cas d’application concret des études que je mène. Il a demandé à l’équipe de nombreuses ressources, et le résultat obtenu et très satisfaisant pour l’ensemble du groupe. Ce projet nous a permis de développer de A à Z un produit complet, répondant, nous l’espérons tous, le mieux possible aux attentes de notre cliente. Le travail en équipe a parfois été délicat à mettre en place, mais malgré des hauts et des bas nous avons tous su trouver les mots pour finalement mener ce projet à bien.
\newpage

\section*{Conclusion}
Tous les membres du groupe ont apprécié travailler sur ce projet. Nous avons tous beaucoup appris et le prototype final est prometteur. Dans un premier temps, nous espérons qu'il sera utilisé pour les séances d'entraînements. Comme le développement est fait pour que ce soit facilement repris et amélioré, nous espérons aussi que cette Machine à Temps de Vol soit améliorée. Ce projet marque une première étape de conception. Ce prototype a de grandes capacités, mais  va-t-il rendre les Machines à Temps de Vol plus accessible ?
\newpage

\end{document}