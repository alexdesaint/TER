\section{Cahier des charges}
Pour la réalisation du projet, nous sommes partis de rien. Il a fallu donc établir un cahier des charges afin de bien définir les besoins du client. Nous nous sommes donc rendus sur place au gymnase de Rangueil pour rencontrer le client et prendre toutes les informations nécessaires et voir les types de trampoline que nous allions utiliser. Notre cahier des charges s’est donc composé au fur et à mesure de trois parties. 

Une partie concernant la machine à temps de vol qui devait respecter les règles de compétition de trampoline émises par la fédération française de trampoline et qui contenait :
\begin{itemize}
    \item Un chronomètre précis au millième de secondes près
    \item Une led qui s’allume chaque fois que le trampoliniste arrive sur la toile pour qu’un juge vérifie que le produit fonctionne correctement
    \item Un bouton permettant de démarrer le chronomètre et enregistrer donc les dix valeurs de sauts
\end{itemize}

La deuxième partie concernait les contraintes émises par le client dont la principale exigence est de réaliser un composant à moindre coût comparé à ceux proposé par le Canada et la Russie (seuls fabricants de machine à temps de vol pour trampoline). Ensuite il fallait un produit transportable, qui puisse s’installer rapidement sur un trampoline et changer de trampoline à tout moment. De plus, le client a demandé à avoir un récapitulatif des différents chronométrages sur Excel à la fin de chaque saut. 

Après l’énoncé des contraintes, nous avons pu proposer quelques fonctionnalités supplémentaires comme l’ajout d’un écran sur le produit pour pouvoir voir pendant les sauts de l’athlète le chronomètre et la réalisation d’un site web dans lequel il y aurait les derniers enregistrements des sauts effectués.
